\documentclass[12pt]{article}
\usepackage[margin=1in]{geometry}
\usepackage[T1]{fontenc}
\usepackage[USenglish]{babel}
\usepackage[nodayofweek,level]{datetime}
\usepackage{amsfonts}
\usepackage{amsmath}
\usepackage{amssymb}
\usepackage{tikz}
\usetikzlibrary{intersections,arrows.meta}
\usepackage{pgfplots}
\usepackage[scr]{rsfso}
\usepackage{array}
\usepackage{stackengine}
\stackMath
\hbadness=10001 %gets rid of "\hfill underfull" warning
\usepackage{tikz,pgfplots}

\pgfplotsset{compat=1.10}
% Uncomment to use "fillbetween" function
% otherwise leave commented because
% the red highlighting is annoying
%\usepgfplotslibrary{fillbetween}
\iffalse
\fi

% Change for each new hw week
\newcommand{\dueDate}{\formatdate{17}{11}{2017}} % day/month/year
\newcommand{\hwNum}{7}


\begin{document}
	
\selectlanguage{USenglish}	
%------------------------ Title Code ------------------------
\title{Homework: Week \hwNum}
\author{Joseph Ismailyan}
\date{}
\maketitle
\begin{flushleft}
Math 100 \\
Due: \dueDate \\ 
Professor Boltje \\
MWF 9:20a-10:25a
\end{flushleft}


%------------------------ Begin Page 1 ------------------------
\begin{minipage}[t]{0.40\textwidth}



\section*{Chapter 9}
\subsection*{12}
\textbf{Proposition}:  \\If $a,b,c\in \mathbb{N} $ and $ab, bc$ and $ac$ all have the same parity, then $a$, $b$, and $c$ all have the same parity.
\newline\textit{Proof.} Suppose $a,b,c\in \mathbb{N} $ and $ab, bc$ and $ac$ all have the same parity. Then we can use the following example: $a=2$, $b=4$, $c=3$. We know that the product of any two natural numbers with opposite parity will be even, so $ab, bc$ and $ac$ are all even. But $a$ , $b$ and $c$ do not all have the same parity.

\subsection*{14}
\textbf{Proposition}:  \\If $A$ and $B$ are sets, then $\mathscr{P}{(A)}\cap\mathscr{P}{(B)}=\mathscr{P}{(A\cap B)}$.
\newline\textit{Proof.} Let $ Y\in\mathscr{P}{(A\cap B)}$. Then $ Y\subseteq(A\cap B) $, so $ Y\subseteq A$ and $ Y\subseteq B$, and by definition of a power set, $ Y\subseteq\mathscr{P}{(A)}$ and $ Y\subseteq\mathscr{P}{(B)}\rightarrow Y\subseteq\mathscr{P}{(A)}\cap\mathscr{P}{(B)}$. Now let $ Y\subseteq\mathscr{P}{(A)}$ and $ Y\subseteq\mathscr{P}{(B)} $. So $ Y\in A$ and $ Y\in B \rightarrow Y\in(A\cap B)\rightarrow Y\subseteq\mathscr{P}{(A\cap B)}$. So $\mathscr{P}{(A)}\cap\mathscr{P}{(B)}=\mathscr{P}{(A\cap B)}$.


\end{minipage}
% Creates vertical line
\hfill\vline\hfill
\begin{minipage}[t]{0.45\textwidth}

\subsection*{18}
\textbf{Proposition}:  \\If $ a,b, c\in\mathbb{N} $, then at least one of $  a-b $, $ a+c $ and $ b-c $ is even.
\newline\textit{Proof.} Suppose $ a,b, c\in\mathbb{N} $. The in order for $  a-b $, $ a+c $,  $ b-c $ to all have odd outcomes, the operands must all have a parity opposite that of the other operand. So then if $ a $ is odd then $ b $ must be even, but $ c $ must be even to make $ a+c $ odd, but then $ c $ must be odd to make $ b-c $ odd. So we can see that $ c $ must simultaneously have two parities, which would be the case no matter what parity $ a $ and $ b $ are. It's impossible for $ a,b,$ and $ c $ to have opposite parities. Therefore, at least one of  $  a-b $, $ a+c $ and $ b-c $ is even.

\subsection*{28}
\textbf{Proposition}:  \\Suppose $ a,b\in\mathbb{Z} $. If $ a|b $ and $ b|a $ then $ a=b $.
\newline\textit{Proof.} Let $ a=1 $ and $ b=-1 $, then $ b=an, n=-1\in\mathbb{Z}$ so $ a|b $, alternatively $ a=bm, m=1\in\mathbb{Z}$ so $ b|a $, but $ a\not=b $.  


\end{minipage}
\pagebreak

%------------------------ End Page 1 ------------------------



%------------------------ Begin Page 2 ------------------------

\begin{minipage}[t]{0.40\textwidth}

\subsection*{34}
\textbf{Proposition}:  \\If $ X\subseteq A\cup B $, then $ X\subseteq A $ or $ X\subseteq B $.
\newline\textit{Proof.} Let $ A=\{1,2,3\}, B=\{3,4,5\}, X=\{1,4\} $. Then $ X\subseteq(A\cup B) $ but $ X\not\subseteq  A$ and $ X\not\subseteq  B$.



\end{minipage}
% Creates vertical line

\begin{minipage}[t]{0.45\textwidth}

\end{minipage}
\pagebreak

\iffalse %delete to add pages 2 and 3
%------------------------ End Page 2 ------------------------

%------------------------ Begin Page 3 ------------------------

\begin{minipage}[t]{0.40\textwidth}
	
	
	
\section*{Section }
\subsection*{}
	

	
\end{minipage}
% Creates vertical line
\hfill\vline\hfill
\begin{minipage}[t]{0.45\textwidth}
	


\section*{Section }
\subsection*{}
	
	
\end{minipage}
\pagebreak

%------------------------ End Page 4 ------------------------

\fi %delete to add pages 2 and 3


\end{document}

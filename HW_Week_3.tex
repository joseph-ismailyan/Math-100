\documentclass[12pt]{article}
\usepackage[margin=1in]{geometry}
\usepackage[T1]{fontenc}
\usepackage[USenglish]{babel}
\usepackage[nodayofweek,level]{datetime}
\usepackage{amsfonts}
\usepackage{amsmath}
\usepackage{amssymb}
\usepackage{tikz}
\usetikzlibrary{intersections,arrows.meta}
\usepackage{pgfplots}
\usepackage[scr]{rsfso}
\usepackage{array}
\usepackage{stackengine}
\stackMath

\usepackage{tikz,pgfplots}

\pgfplotsset{compat=1.10}
% Uncomment to use %fillbetween" function
% otherwise leave commented because
% the red highlighting is annoying
%\usepgfplotslibrary{fillbetween}

% Change due date below
\newcommand{\dueDate}{\formatdate{13}{10}{2017}}


\begin{document}
	
%\selectlanguage{USenglish}	

\title{Homework: Week 3}
\author{Joseph Ismailyan}
\date{}
\maketitle
\begin{flushleft}
Math 100 \\
Due: \dueDate \\ 
Professor Boltje \\
MWF 9:20a-10:25a
\end{flushleft}

% Insert first section here
% Minipage creates the columns
\begin{minipage}[t]{0.40\textwidth}

\section*{Section 2.7}
\subsection*{6.}
Translate to English: \\$  \exists n \in \mathbb{N}, \forall X \in \mathscr{P}(\mathbb{N}) , |X| < n$\\

Translated: There exists a natural number $ n $ for every subset $ X $ of $ \mathbb{N} $ where $ |X| <n $.\\

This can be proven false with a simple counter example as follows: \\ $ X=\{1,2\} \text{ and } n=1 \text{ then } |X| > n$


\section*{Section 2.9}
\subsection*{6.}
Translate to SL:\\
For every positive number $ \varepsilon $ there is a positive number $ M $ for which \\$ |f(x)-b|<\varepsilon $, whenever $ x>M $.\\

Logic:

$\forall \varepsilon\in\mathbb{R},\varepsilon\geq 0, \exists M\in\mathbb{R},M\geq 0,$
\\
$
(x>M) \implies (|f(x)-b|<\varepsilon)
$

\end{minipage}
% Creates verticle line
\hfill\vline\hfill
\begin{minipage}[t]{0.45\textwidth}

\section*{Section 2.10}
\subsection*{8.}
Translate: If $ x $ is a rational number and $ x \neq 0 $, then $ \tan(x) $ is not a rational number.\\

SL: $ (x\in \mathbb{Q} \land x\neq 0)  \implies (tan(x) \notin \mathbb{Q})$ \\
Negation:$ x\in \mathbb{Q} \land x\neq 0 \land tan(x) \in \mathbb{Q}$ \\

Translation (Answer):  $ x $ is a rational number and $ x \neq 0 $ and $ \tan(x) $ is a rational number.



\section*{Section 3.1}
\subsection*{4.}
Question: Five cards are dealt off of a standard 52-card deck and lined up in a row. How
many such line-ups are there in which all 5 cards are of the same suit?

Answer: $ 13*12*11*10*9 = 154,440 $
\subsection*{12.}
Answer: $ 5^5-(4^5+(5*4^4))=821$

\end{minipage}
\pagebreak


\begin{minipage}[t]{0.40\textwidth}

\section*{Section 3.2}
\subsection*{8.}
Answer: There are 4 places the odd sequence could start.\\
$ O = \text{odd}, E = \text{even} $

$$ \stackon{4}{O}\stackon{3}{O}\stackon{2}{O}\stackon{1}{O}\stackon{3}{E}\stackon{2}{E}\stackon{1}{E}  = \stackunder{144}{+}
$$
$$ \stackon{3}{E}\stackon{6}{O}\stackon{3}{O}\stackon{2}{O}\stackon{1}{O}\stackon{2}{E}\stackon{1}{E}  = \stackunder{144}{+}
$$
$$ \stackon{3}{E}\stackon{2}{E}\stackon{4}{O}\stackon{3}{O}\stackon{2}{O}\stackon{1}{O}\stackon{1}{E}  = \stackunder{144}{+}
$$
$$ \stackon{3}{E}\stackon{2}{E}\stackon{1}{E}\stackon{4}{O}\stackon{3}{O}\stackon{2}{O}\stackon{1}{O}  = 144
$$
$ \text{Total}: 576 $

\section*{Section 3.3}
\subsection*{10.}
Answer: $ \binom{5}{3}\binom{7}{2} = 210$
\subsection*{12.}
Answer: \\$ \binom{21}{10}\binom{11}{11} =352,716$

\end{minipage}
% Creates verticle line
\hfill\vline\hfill
\begin{minipage}[t]{0.45\textwidth}
\section*{Section 3.4}
\subsection*{12.}
Show that:
\[
\binom{n}{k}\binom{k}{m}=\binom{n}{m}\binom{n-m}{k-m}
\]
LHS:\\
\[=\frac{n!}{k!(n-k)!}*\frac{k!}{m!(k-m)!}\]
\[=\frac{n!}{m!(n-k)!(k-m)!}\]
RHS:\\
\[=\frac{n!}{m!(n-m)!}*\frac{n-m!}{(k-m)![(n-m)-(k-m)]!}\]
\[=\frac{n!}{m!(k-m)![n-m-k+m]!}\]
\[=\frac{n!}{m!(k-m)!(n-k)!}\]

\[\frac{n!}{m!(n-k)!(k-m)!} = \frac{n!}{m!(k-m)!(n-k)!}\]

\[\therefore \binom{n}{k}\binom{k}{m}=\binom{n}{m}\binom{n-m}{k-m}\]

\end{minipage}
\pagebreak


\end{document}